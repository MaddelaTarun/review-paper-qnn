\section{Discussion}

The experimental findings indicate that QNN-based classifiers offer promising robustness against adversarial attacks compared to classical models. This behaviour may stem from quantum state properties such as superposition and entanglement, which provide richer representational capacity and smoother classification boundaries. Although the QNN does not surpass transformer models in raw accuracy, its superior robustness under perturbations positions it as a valuable direction for secure misinformation detection.

Despite these strengths, quantum computing remains constrained by hardware limitations, shallow circuits, and simulator overhead. The current work relies on classical simulation rather than execution on real quantum hardware. As quantum processors mature, deeper and more expressive quantum circuits may further improve accuracy and robustness.

Future work includes exploring multimodal quantum models integrating text and images, advanced adversarial defense strategies, and deployment on quantum hardware for empirical validation.
