\section{Introduction}

The widespread circulation of misleading or fabricated news content has become a critical societal challenge, with significant consequences for public opinion, political discourse, and information reliability. Digital platforms enable rapid dissemination of unverified content, making the early and accurate detection of fake news an essential research priority. Automated detection systems based on classical machine learning and deep neural networks have achieved considerable progress; however, their robustness against adversarial manipulation remains limited. Small, semantically equivalent modifications to textual inputs can cause misclassification, exposing serious vulnerabilities in real-world scenarios.

Recent advances in quantum computing have opened new opportunities for the development of quantum-enhanced machine learning models. Quantum Neural Networks (QNNs), leveraging high-dimensional Hilbert space representations and quantum entanglement, offer theoretical advantages in expressivity and feature encoding. While several studies demonstrate the feasibility of QNNs for text classification tasks, the adversarial robustness of such models in fake news detection remains largely unexplored.

This research addresses this gap by designing a QNN-based fake news classification framework and evaluating its behaviour under adversarial attacks commonly used in natural language processing. The contributions of this work are fourfold:  
\begin{enumerate}
    \item Development of a hybrid classical--quantum model integrating classical embeddings with quantum circuits for classification.
    \item Generation of adversarial samples using synonym substitution and semantic-preserving transformations.
    \item Comparative robustness evaluation of QNNs against traditional machine learning and deep learning baselines.
    \item Introduction of adversarial training strategies to enhance the robustness of the quantum model.
\end{enumerate}

The remainder of this paper is structured as follows. Section~2 reviews relevant literature. Section~3 details the methodology, including data preprocessing, quantum encoding, and model architecture. Section~4 outlines the experimental setup. Section~5 presents results, followed by discussion in Section~6. Section~7 concludes the paper and highlights future research directions.
