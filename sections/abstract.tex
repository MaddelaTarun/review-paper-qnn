\section*{Abstract}

The rapid spread of misinformation across digital platforms has created an urgent need for reliable and resilient fake news detection systems. Classical deep learning models demonstrate strong performance on benchmark datasets, yet they remain vulnerable to adversarial perturbations that subtly alter textual content while preserving semantic meaning. Recent developments in quantum machine learning introduce Quantum Neural Networks (QNNs) as promising alternatives due to their expressive Hilbert-space representations and non-classical correlations. However, their adversarial robustness in fake news detection remains largely unexplored.  
This study presents a hybrid QNN-based framework trained on the WELFake dataset and evaluates its performance under adversarial conditions generated through synonym-level and semantic-preserving attacks. Experimental results indicate that QNNs achieve competitive baseline accuracy and exhibit improved stability under perturbation, demonstrating up to a 9\% increase in robustness compared to classical counterparts. These findings highlight the potential of quantum-enhanced models in developing secure and trustworthy misinformation detection systems and provide future directions for advancing adversarially robust quantum text classifiers.
